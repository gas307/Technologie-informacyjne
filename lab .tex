\documentclass{article}
\usepackage[utf8]{inputenc}
\usepackage{graphicx}	
\usepackage{longtable}
\usepackage{array}
\usepackage{pgfplots}
\pgfplotsset{compat = newest}

\title{Spadek swobodny}
\author{Dominik Gasztych}
\date{January 2023}

\begin{document}

\maketitle

\section{Wprowadzenie teoretyczne}
Spadek swobodny – ruch odbywający się wyłącznie pod
wpływem ciężaru (siły grawitacji), bez oporów ośrodka. Przyjmuje się, że spadek
rozpoczyna się od spoczynku, w odróżnieniu od ruchu w polu grawitacyjnym z
prędkością początkową zwanego rzutem.
\begin{equation}
    h(t)=h_0- gt^2/2
\end{equation}

\section{Przebieg eksperymentu}

\begin{center}
\includegraphics[scale=0.5]{rys0010.jpg}
\end{center}

\section{Wyniki pomiaru}

\begin{center}
\begin{longtable}{|c|c|}
\caption{Czas i droga w spadku swobodnym} \\

\hline
t{[}s{]} & s{[}m{]} \\
0,0      & 0,00     \\
0,1      & 0,05     \\
0,2      & 0,20     \\
0,3      & 0,44     \\
0,4      & 0,78     \\
0,5      & 1,23     \\
0,6      & 1,77     \\
0,7      & 2,40     \\
0,8      & 3,14     \\
0,9      & 3,97     \\
1,0      & 4,91     \\
1,1      & 5,94     \\
1,2      & 7,06     \\
1,3      & 8,29     \\
1,4      & 9,61     \\
1,5      & 11,04    \\
1,6      & 12,56    \\
1,7      & 14,18    \\
1,8      & 15,89    \\
1,9      & 17,71    \\
2,0      & 19,62    \\
2,1      & 21,63    \\
2,2      & 23,74    \\
2,3      & 25,95    \\
2,4      & 28,25    \\
2,5      & 30,66    \\
2,6      & 33,16    \\
2,7      & 35,76    \\
2,8      & 38,46    \\
2,9      & 41,25    \\
3,0      & 44,15    \\
3,1      & 47,14    \\
3,2      & 50,23    \\
3,3      & 53,42    \\
3,4      & 56,70    \\
3,5      & 60,09    \\
3,6      & 63,57    \\
3,7      & 67,15    \\
3,8      & 70,83    \\
3,9      & 74,61    \\
4,0      & 78,48    \\
4,1      & 82,45    \\
4,2      & 86,52    \\
4,3      & 90,69    \\
4,4      & 94,96    \\
4,5      & 99,33    \\
4,6      & 103,79   \\
4,7      & 108,35   \\
4,8      & 113,01   \\
4,9      & 117,77   \\
5,0      & 122,63   \\
5,1      & 127,58   \\
5,2      & 132,63   \\
5,3      & 137,78   \\
5,4      & 143,03   \\
5,5      & 148,38   \\
5,6      & 153,82   \\
5,7      & 159,36   \\
5,8      & 165,00   \\
5,9      & 170,74   \\
6,0      & 176,58   \\
6,1      & 182,52   \\
6,2      & 188,55   \\
6,3      & 194,68   \\
6,4      & 200,91   \\
6,5      & 207,24   \\
6,6      & 213,66   \\
6,7      & 220,19   \\
6,8      & 226,81   \\
6,9      & 233,53   \\
7,0      & 240,35   \\
7,1      & 247,26   \\
7,2      & 254,28   \\
7,3      & 261,39   \\
7,4      & 268,60   \\
7,5      & 275,91   \\
7,6      & 283,31   \\
7,7      & 290,82   \\
7,8      & 298,42   \\
7,9      & 306,12   \\
8,0      & 313,92   \\
8,1      & 321,82   \\
8,2      & 329,81   \\
8,3      & 337,91   \\
8,4      & 346,10   \\
8,5      & 354,39   \\
8,6      & 362,77   \\
8,7      & 371,26   \\
8,8      & 379,84   \\
8,9      & 388,53   \\
9,0      & 397,31   \\
9,1      & 406,18   \\
9,2      & 415,16   \\
9,3      & 424,23   \\
9,4      & 433,41   \\
9,5      & 442,68   \\
9,6      & 452,04   \\
9,7      & 461,51   \\
9,8      & 471,08   \\
9,9      & 480,74   \\
10,0     & 490,50  
\end{longtable}
\end{center}


\section{Wnioski}
Przyspieszenie spadających ciał nie zależy od ich masy. Spadek swobodny, to taki w
którym nie ma oporu powietrza  - ciała puszczone z tej samej wysokości  niezależnie
od masy i kształtu spadają w tym samym czasie pod wpływem siły grawitacji. Czas
spadku zależy od wysokości.
 \\ 

 \begin{center}
\begin{tikzpicture}
\begin{axis}[
    title={Droga i czas w spadku swobodnym},
    xlabel={T[s]},
    ylabel={S[m]},
    xmin=0, xmax=10,
    ymin=0, ymax=600,
    legend pos=north west,
    ymajorgrids=true,
    grid style=dashed,
]

\addplot+[
    only marks,
    scatter,
    mark=halfcircle*,
    mark size=1pt]
table[meta = s]
{dane3.txt};

\addplot [
    domain=0:10, 
    samples=100, 
    color=red,
]
{(9.8*x*x)/2};

 
\end{axis}
\end{tikzpicture}
\end{center}

\end{document}
