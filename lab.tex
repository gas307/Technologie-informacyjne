\documentclass{article}
\usepackage[utf8]{inputenc}
\usepackage{graphicx}
\graphicspath{ {rys0010.jpg} }

\title{Spadek swobodny}
\author{Dominik Gasztych}
\date{Wrocław 13.12.2022}

\begin{document}

\maketitle

\section{Opis teoretyczny}
    Spadek swobodny – ruch odbywający się wyłącznie pod wpływem ciężaru (siły grawitacji), bez oporów ośrodka. Przyjmuje się, że spadek rozpoczyna się od spoczynku, w odróżnieniu od ruchu w polu grawitacyjnym z prędkością początkową zwanego rzutem.

\begin{equation}
\label{równanie}
t & = \sqrt{\frac{2 h}{g}}
\end{equation}
\section{Przebieg doświadczenia}
\begin{figure}
    \centering
    \includegraphics{rys0010.jpg}
    \caption{Rys.1}
    \label{fig:schemat doświadczenia}
\end{figure}
\section{Wyniki}
\begin{tabular}{|l|c|r|}
    \hline
    lp  &   t[s]  &   s[m]\\
    \hline
    1   &    0.1   &   0.05\\
    \hline
    2   &    0.2   &   0.2\\
    \hline
    3   &    0.3   &   0.44\\
    \hline
\end{tabular}
\section{Wnioski}
Przyspieszenie spadających ciał nie zależy od ich masy. Spadek swobodny, to taki w którym nie ma oporu powietrza  - ciała puszczone z tej samej wysokości  niezależnie od masy i kształtu spadają w tym samym czasie pod wpływem siły grawitacji. Czas spadku zależy od wysokości.
\section{Spis treści}
strona 1
\newline
Wprowadzenie teoretyczne
\vspace.
Przebieg eksperymentu
\newline
strona 2
\newline
wyniki pomiarów
\newline
wnioski


\end{document}
